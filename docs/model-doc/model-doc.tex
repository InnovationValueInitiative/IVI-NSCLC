\documentclass[11pt,final,fleqn]{article}\usepackage[]{graphicx}\usepackage[]{color}
%% maxwidth is the original width if it is less than linewidth
%
%% otherwise use linewidth (to make sure the graphics do not exceed the margin)
\makeatletter
\def\maxwidth{ %
  \ifdim\Gin@nat@width>\linewidth
    \linewidth
  \else
    \Gin@nat@width
  \fi
}
\makeatother

\definecolor{fgcolor}{rgb}{0.345, 0.345, 0.345}
\newcommand{\hlnum}[1]{\textcolor[rgb]{0.686,0.059,0.569}{#1}}%
\newcommand{\hlstr}[1]{\textcolor[rgb]{0.192,0.494,0.8}{#1}}%
\newcommand{\hlcom}[1]{\textcolor[rgb]{0.678,0.584,0.686}{\textit{#1}}}%
\newcommand{\hlopt}[1]{\textcolor[rgb]{0,0,0}{#1}}%
\newcommand{\hlstd}[1]{\textcolor[rgb]{0.345,0.345,0.345}{#1}}%
\newcommand{\hlkwa}[1]{\textcolor[rgb]{0.161,0.373,0.58}{\textbf{#1}}}%
\newcommand{\hlkwb}[1]{\textcolor[rgb]{0.69,0.353,0.396}{#1}}%
\newcommand{\hlkwc}[1]{\textcolor[rgb]{0.333,0.667,0.333}{#1}}%
\newcommand{\hlkwd}[1]{\textcolor[rgb]{0.737,0.353,0.396}{\textbf{#1}}}%
\let\hlipl\hlkwb

\usepackage{framed}
\makeatletter
\newenvironment{kframe}{%
 \def\at@end@of@kframe{}%
 \ifinner\ifhmode%
  \def\at@end@of@kframe{\end{minipage}}%
  \begin{minipage}{\columnwidth}%
 \fi\fi%
 \def\FrameCommand##1{\hskip\@totalleftmargin \hskip-\fboxsep
 \colorbox{shadecolor}{##1}\hskip-\fboxsep
     % There is no \\@totalrightmargin, so:
     \hskip-\linewidth \hskip-\@totalleftmargin \hskip\columnwidth}%
 \MakeFramed {\advance\hsize-\width
   \@totalleftmargin\z@ \linewidth\hsize
   \@setminipage}}%
 {\par\unskip\endMakeFramed%
 \at@end@of@kframe}
\makeatother

\definecolor{shadecolor}{rgb}{.97, .97, .97}
\definecolor{messagecolor}{rgb}{0, 0, 0}
\definecolor{warningcolor}{rgb}{1, 0, 1}
\definecolor{errorcolor}{rgb}{1, 0, 0}
\newenvironment{knitrout}{}{} % an empty environment to be redefined in TeX

\usepackage{alltt}

% basic packages
\usepackage[T1]{fontenc}
\usepackage[margin=1in] { geometry }
\usepackage{amssymb,amsmath, bm}
\usepackage{verbatim}
\usepackage[latin1]{inputenc}
%\usepackage[OT1]{fontenc}
\usepackage{setspace}
\usepackage{natbib}
\usepackage{enumitem}
\usepackage[hyphens,spaces,obeyspaces]{url}
\usepackage[font={bf}]{caption}
%\usepackage{pgfplots}
%\usepackage[font={bf}]{caption}
\usepackage{latexsym}
%\usepackage{euscript}
\usepackage{graphicx}
\usepackage{marvosym}
%\usepackage[varg]{txfonts}  Older version of ``g'' in math.
\usepackage{pdflscape}
\usepackage{algorithm}

% bibliography packages
\usepackage{natbib}
\bibpunct{(}{)}{;}{a}{}{,}
\bibliographystyle{apa}
\renewcommand{\bibname}{References}

% hyperref options
\usepackage{color}
\usepackage{hyperref}
\usepackage{xcolor}
\hypersetup{
    colorlinks,
    linkcolor={blue!50!black},
    citecolor={blue!50!black},
    urlcolor={blue!80!black}
}
\newcommand*{\Appendixautorefname}{Appendix}
\renewcommand*{\sectionautorefname}{Section}
\renewcommand*{\subsectionautorefname}{Section}
\renewcommand*{\subsubsectionautorefname}{Section}
\newcommand{\subfigureautorefname}{\figureautorefname}
\newcommand{\aref}[1]{\hyperref[#1]{Appendix~\ref{#1}}}
\newcommand{\algorithmautorefname}{Algorithm}

% packages for tables
\usepackage{longtable}
\usepackage{booktabs, threeparttable}
\usepackage{threeparttablex}
\usepackage{tabularx}
% dcolumn package
\usepackage{dcolumn}
\newcolumntype{.}{D{.}{.}{-1}}
\newcolumntype{d}[1]{D{.}{.}{#1}}
\captionsetup{belowskip=10pt,aboveskip=-5pt}
\usepackage{multirow}
% rotating package
\usepackage[figuresright]{rotating}
\usepackage{pdflscape}
\usepackage{subcaption}
\usepackage{caption} 
\captionsetup[table]{skip=5pt}

% packages for figures
\usepackage{grffile}
\usepackage{afterpage}
\usepackage{float}
\usepackage[section]{placeins}
\usepackage[export]{adjustbox}

% theorem package
\usepackage{theorem}
\theoremstyle{plain}
\theoremheaderfont{\scshape}
\newtheorem{theorem}{Theorem}
\newtheorem{assumption}{Assumption}
\newtheorem{lemma}{Lemma}
\newtheorem{proposition}{Proposition}
\newtheorem{remark}{Remark}
\newcommand{\qed}{\hfill \ensuremath{\Box}}
\newcommand\indep{\protect\mathpalette{\protect\independenT}{\perp}}
\DeclareMathOperator{\sgn}{sgn}
\DeclareMathOperator{\tr}{tr}
\DeclareMathOperator{\argmin}{arg\min}
\DeclareMathOperator{\argmax}{arg\max}
\def\independenT#1#2{\mathrel{\rlap{$#1#2$}\mkern2mu{#1#2}}}
\providecommand{\norm}[1]{\lVert#1\rVert}
\renewcommand\r{\right}
\renewcommand\l{\left}
\newcommand\E{\mathbb{E}}
\newcommand\dist{\buildrel\rm d\over\sim}
\newcommand\iid{\stackrel{\rm i.i.d.}{\sim}}
\newcommand\ind{\stackrel{\rm indep.}{\sim}}
\newcommand\cov{{\rm Cov}}
\newcommand\var{{\rm Var}}
\newcommand\SD{{\rm SD}}
\newcommand\bone{\mathbf{1}}
\newcommand\bzero{\mathbf{0}}
\DeclareMathOperator{\logit}{logit}
\DeclareMathOperator{\Cat}{Cat}
\DeclareMathOperator{\Multinomial}{Multinomial}

% file paths and definitions
\makeatletter
\newcommand*\ExpandableInput[1]{\@@input#1 }
\makeatother

% spacing 
\usepackage[compact]{titlesec}
\setlength{\parindent}{0pt}
\setlength{\parskip}{6pt plus 2pt minus 1pt}
\setstretch{1}

% appendix settings
\usepackage[toc,page,header]{appendix}
\renewcommand{\appendixpagename}{\centering Appendices}
\usepackage{chngcntr}
\usepackage{etoolbox}
\usepackage{lipsum}

% new commands
\newcommand\CPP{{C\texttt{++}}}
\newcommand\R{{\textsf{R}}}
\newcommand*\sameaff[1][\value{footnote}]{\footnotemark[#1]}
\newcommand\floor[1]{\lfloor#1\rfloor}
\newcommand\ceil[1]{\lceil#1\rceil}
\newcommand{\code}[1]{\texttt{#1}}

% subsubsection

\titleclass{\subsubsubsection}{straight}[\subsection]

\newcounter{subsubsubsection}[subsubsection]
\renewcommand\thesubsubsubsection{\thesubsubsection.\arabic{subsubsubsection}}
\renewcommand\theparagraph{\thesubsubsubsection.\arabic{paragraph}} % optional; useful if paragraphs are to be numbered

\titleformat{\subsubsubsection}
  {\normalfont\normalsize\bfseries}{\thesubsubsubsection}{1em}{}
\titlespacing*{\subsubsubsection}
{0pt}{3.25ex plus 1ex minus .2ex}{1.5ex plus .2ex}

\makeatletter
\renewcommand\paragraph{\@startsection{paragraph}{5}{\z@}%
  {3.25ex \@plus1ex \@minus.2ex}%
  {-1em}%
  {\normalfont\normalsize\bfseries}}
\renewcommand\subparagraph{\@startsection{subparagraph}{6}{\parindent}%
  {3.25ex \@plus1ex \@minus .2ex}%
  {-1em}%
  {\normalfont\normalsize\bfseries}}
\def\toclevel@subsubsubsection{4}
\def\toclevel@paragraph{5}
\def\toclevel@paragraph{6}
\def\l@subsubsubsection{\@dottedtocline{4}{7em}{4em}}
\def\l@paragraph{\@dottedtocline{5}{10em}{5em}}
\def\l@subparagraph{\@dottedtocline{6}{14em}{6em}}
\makeatother

\setcounter{secnumdepth}{4}
\setcounter{tocdepth}{4}

% title
\title{A Description of the IVI-NSCLC Model v1.0} 
\author{Devin Incerti\footnote{\href{http://www.thevalueinitiative.org/}{Innovation and Value Initiative}} \and Jeroen P. Jansen\sameaff}
\date{\today}
\begin{document}
\maketitle

\begingroup
 \hypersetup{linkcolor=black} \tableofcontents
 \listoffigures
 \listoftables
\endgroup

\clearpage
\phantomsection
\section{Open Source Value Project}\label{sec:osvp}

\section{Purpose}\label{sec:purpose}

\section{Components}\label{sec:components}

\section{Model structure}\label{sec:modstruct}
\subsection{Disease model}\label{subsec:modstruct-disease}
\subsection{Adverse events}\label{subsec:modstruct-ae}
\subsection{Cost and utility}\label{subsec:modstruct-statevals}
\subsection{Patient heterogeneity}\label{subsec:modstruct-heterogeneity}
\subsection{Rationale for individual-level simulation}\label{subsec:modstruct-indivsim}

\section{Model outcomes}\label{sec:modout}
\subsection{Health outcomes}\label{subsec:modout-health}
\subsection{Risks}\label{subsec:modout-risks}
\subsection{Costs}\label{subsec:modout-costs}
\subsection{Value assessment}\label{subsec:modout-value}

\section{Simulation and uncertainty analysis}\label{sec:sim}
\subsection{Parameter uncertainty}\label{subsec:sim-paramameter-uncertainty}
\citet{baio2015probabilistic}
\subsection{Structural uncertainty}\label{subsec:sim-structural-uncertainty}
\subsection{Implementation}\label{subsec:sim-implementation}

\section{Source data and parameter estimation}\label{sec:data}
Key parameters for the model relate to: (i) transition probabilities; (ii) adverse events; (iii) utilities; (iv) healthcare resource use; and (v) productivity. Parameter estimates are based on currently available published evidence identified by means of a systematic literature review (SLR) and synthesized with meta-analysis techniques where appropriate. Details of our SLR (\autoref{app:slr}) and NMA (\autoref{app:nma})  techniques are provided in the Appendix. 

\subsection{Transition probabilities}\label{subsec:data-transprobs}
Protocol Section 7.4.2.5.1 to go here.

\subsection{Adverse events}\label{subsec:data-aes}
Since adverse events in nearly all clinical trials were reported as the number of patients experiencing the event, the NMA was performed on the proportion of patients experiecning the event of interest with a binomial likelihood and logit link \citep[Chapter~2]{dias2018network}.

\subsection{Utilities}\label{subsec:data-utility}
To be added.

\subsection{Resource use, productivity, and cost}\label{subsec:data-costs}
To be added.


\begin{appendices}
\setcounter{table}{0}
\renewcommand{\thetable}{A\arabic{table}}
\setcounter{figure}{0}
\renewcommand{\thefigure}{A\arabic{figure}}
\setcounter{equation}{0}
\renewcommand{\theequation}{A\arabic{equation}}

\section{Systematic Literature Review}\label{app:slr}
\subsection{Transition probabilities}\label{app:slr-transprobs}
Protocol 7.4.1.1

\subsection{Adverse events}\label{app:slr-aes}
To be added.

\subsection{Utilities}\label{app:slr-utility}
Protocol 7.4.1.2

\subsection{Resource use, productivity, and costs}\label{app:slr-costs}
Protocol 7.4.1.3

\subsection{Study identification}\label{app:slr-study-identification}
Protocol 7.4.1.4

\subsection{Study selection}\label{app:slr-study-selection}
Protocol 7.4.1.5

\subsection{Data collection}\label{app:slr-data-collection}
Protocol 7.4.1.6

\subsection{Limitations}\label{app:slr-limitations}
Protocol 7.4.1.7

\section{Network meta-analysis}\label{app:nma}
\subsection{Population, interventions, and outcomes of interest}\label{app:nma-pop}
Protocol 7.4.2.1

\subsection{Feasibility assessment}\label{app:nma-feasibility}
Protocol 7.4.2.2

\subsection{Evaluation of consistency between direct and indirect comparisons}\label{app:nma-consistency}
Protocol 7.4.2.3

\subsection{Estimation of relative treatment effects under the assumption of consistency}\label{app:nma-estimation}
Protocol 7.4.2.4

\subsection{Models, likelihood, priors}\label{app:nma-likelihood}
Protocol 7.4.2.5 (but not Protocol 7.4.2.5.1 or 7.4.2.5.2)

\subsection{Model selection}\label{app:nma-model-selection}
Protocol 7.4.2.6

\subsection{Software}\label{app:nma-software}
Protocol 7.4.2.7

\end{appendices}
\pdfbookmark[1]{References}{References}
\bibliography{references}


\end{document}